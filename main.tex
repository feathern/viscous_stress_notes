\documentclass[10pt, letterpaper]{article}
\usepackage{graphicx}   % need this for pngs
\usepackage{epsfig}     % need this for eps files
\usepackage{amsmath}
\usepackage[table]{xcolor}
\usepackage{color}
\setlength{\parindent}{0pt}
\setlength{\parskip}{5pt}
\usepackage{textpos}

% use a larger page size; otherwise, it is difficult to have complete
% code listings and output on a single page
\usepackage{fullpage}


% use the listings package for code snippets. define keywords for prm files
% and for gnuplot
\usepackage{listings}
\lstset{frame=tb,
  language=Fortran,
  aboveskip=3mm,
  belowskip=3mm,
  showstringspaces=false,
  columns=flexible,
  basicstyle={\small\ttfamily},
  numbers=none,
  numberstyle=\tiny\color{gray},
  keywordstyle=\color{blue},
  commentstyle=\color{dkgreen},
  stringstyle=\color{mauve},
  breaklines=true,
  breakatwhitespace=true,
  tabsize=3
}

\lstdefinelanguage{prmfile}{morekeywords={set,subsection,end},
                            morecomment=[l]{\#},escapeinside={\%\%}{\%},}
\lstdefinelanguage{gnuplot}{morekeywords={plot,using,title,with,set,replot},
                            morecomment=[l]{\#},}


% use the hyperref package; set the base for relative links to
% the top-level Rayleigh directory so that we can link to
% files in the Rayleigh tree without having to specify the
% location relative to the directory where the pdf actually
% resides
\usepackage[colorlinks,linkcolor=blue,urlcolor=blue,citecolor=blue,destlabel=true,baseurl=../]{hyperref}

\definecolor{dkgreen}{rgb}{0,0.6,0}
\definecolor{gray}{rgb}{0.5,0.5,0.5}
\definecolor{mauve}{rgb}{0.58,0,0.82}

\newcommand{\rayleigh}{\textsc{Rayleigh}}

\begin{document}

\definecolor{dark_grey}{gray}{0.3}
\definecolor{aspect_blue}{rgb}{0.3125,0.6875,0.9375}

%LINE 1%

\renewcommand{\familydefault}{\sfdefault}

%\pagenumbering{gobble}

\pagenumbering{arabic}
\renewcommand{\abstractname}{Overview}


\title{Notes on Viscous Stress Tensor}
\author{Nick Featherstone}
\date{}
\maketitle

\newcommand{\uvec}{\boldsymbol{u}}
\newcommand{\vgrad}{\boldsymbol{\nabla}}
\newcommand{\Dscript}{\boldsymbol{\mathcal{D}}}
\newcommand{\ebold}{\boldsymbol{e}}
\newcommand{\etwiddle}{\boldsymbol{\tilde{e}}}
\newcommand{\etwid}{\tilde{e}}
\newcommand{\sintheta}{\mathrm{sin}\theta}
\newcommand{\cottheta}{\mathrm{cot}\theta}
\newcommand{\csctheta}{\mathrm{csc}\theta}
\newcommand{\dvec}{\boldsymbol{d}}
\newcommand{\bvec}{\boldsymbol{b}}
\newcommand{\mvec}{\boldsymbol{m}}
\newcommand{\divu}{\vgrad\cdot\uvec}
\newcommand{\half}{\frac{1}{2}}
\newcommand{\third}{\frac{1}{3}}
\section{Overview}
The momentum equation, in the absence of rotation and any body forces is given by
\begin{equation}
\label{eq:momentum1}
\rho\frac{D\uvec}{Dt} = -\vgrad P +\vgrad\cdot\Dscript,
\end{equation}
where $\uvec$ is the velocity vector, $P$ is the pressure, $\rho$ is the density.  The second order viscous stress tensor, which we will define shortly, is given by $\Dscript$.   Our goal is to formulate and expression for $\vgrad\cdot\Dscript$ for the situation where $\rho$ and kinematic viscosity $\nu$ are allowed to vary abitrarily with position.

\section{Viscous Stress Tensor:  Definitions}
The viscous stress tensor $\Dscript$ has elements $\mathcal{D}_{ij}$ defined by
\begin{equation}
\label{eq:Dij}
\mathcal{D}_{ij}=2\rho\nu\left[e_{ij}-\third\left(\divu\right)\delta_{ij}\right],
\end{equation}
where $\delta_{ij}$ is the Kronecker delta, and $e_{ij}$ are the elements of the symmetric, second order strain rate tensor $\ebold$.  The elements of $\ebold$ are defined as follows
\begin{align}
e_{rr}& = \frac{\partial u_r}{\partial r}                                              \\
e_{\theta\theta}& = \frac{1}{r}\frac{\partial u_\theta}{\partial\theta}+\frac{u_r}{r}  \\
e_{\phi\phi}& = \frac{1}{r\sintheta}\frac{\partial u_\phi}{\partial\phi} +\frac{u_r}{r} + \frac{u_\theta\cottheta}{r} \\
e_{r\theta}& = \half\left[ r\frac{\partial}{\partial r}\left( \frac{u_\theta}{r} \right) +  \frac{1}{r}\frac{\partial u_r}{\partial\theta}  \right]    = e_{\theta r}              \\
e_{r\phi}&  = \half\left[ \frac{1}{r\sintheta}\frac{\partial u_r}{\partial\phi} + r\frac{\partial}{\partial r}\left(\frac{u_\phi}{r} \right)   \right]   = e_{\phi r}\\
e_{\theta\phi}& = \half\left[ \frac{\sintheta}{r}\frac{\partial}{\partial\theta}\left( \frac{u_\phi}{\sintheta} \right) +  \frac{1}{r\sintheta}\frac{\partial u_\theta}{\partial\phi}  \right] = e_{\phi\theta}
\end{align}
 
\section{Divergence of the Stress Tensor}
Before proceeding, let's use the dynamic viscosity $\mu\equiv\rho\nu$ and use $\etwiddle$ to denote the combination of the stress tensor and divergence piece.  Namely,
\begin{equation}
\label{eq:etwiddle}
\tilde{e}_{ij} \equiv e_{ij} - \third\left(\divu\right)\delta_{ij},
\end{equation}
so that 
\begin{equation}
\mathcal{D}_{ij} = 2\mu\tilde{e}_{ij}.
\end{equation}
Once can then show (on some notes I lost long ago, but people do this all the time) that
\begin{equation}
\vgrad\cdot\Dscript = \mu\left[\nabla^2\uvec + \third\vgrad\left(\divu\right)  \right] + 2\left(\vgrad\mu\right)\cdot\etwiddle.
\end{equation}

The term involving $\nabla^2$ can be looked up easily enough.   Our job is to formulate expressions for the second and third terms.

\section{Gradient of $\vgrad\cdot\uvec$}
Let $\dvec\equiv\vgrad\left(\divu\right)$.   When then have that
\begin{align}
d_r& =  \frac{1}{r} \left[2 \frac{\partial u_r}{\partial r} + \frac{\partial}{\partial r} \left(\frac{\partial u_{\theta}}{\partial \theta} \right) + \cottheta \frac{\partial u_{\theta}}{\partial r} + \csctheta \frac{\partial}{\partial r} \left(\frac{\partial u_{\phi}}{\partial \phi} \right) \right] - \frac{1}{r^2} \left[2 u_r + \frac{\partial u_{\theta}}{\partial \theta} + \cottheta u_{\theta} + \csctheta \frac{\partial u_{\phi}}{\partial \phi} \right] + \frac{\partial^2 u_r}{\partial r^2}        \\
d_\theta& = \frac{1}{r^2} \left[\frac{\partial^2 u_{\theta}}{\partial \theta^2} + 2 \frac{\partial u_r}{\partial \theta} - u_{\theta} + \cottheta \frac{\partial u_{\theta}}{\partial \theta} + \csctheta \frac{\partial}{\partial \theta} \left(\frac{\partial u_{\phi}}{\partial \phi} \right) - \csctheta^2 u_{\theta} - \cottheta \csc\theta \frac{\partial u_{\phi}}{\partial \phi} \right] + \frac{1}{r} \frac{\partial}{\partial \theta} \left(\frac{\partial u_r}{\partial r} \right)     \\
d_\phi& = \frac{1}{r^2} \csctheta \left[2 \frac{\partial u_r}{\partial \phi} + \frac{\partial}{\partial \phi} \left(\frac{\partial u_{\theta}}{\partial \theta} \right) + \cottheta \frac{\partial u_{\theta}}{\partial \phi} + \csctheta \frac{\partial^2 u_{\phi}}{\partial \phi^2} \right] + \frac{1}{r} \csctheta \frac{\partial}{\partial \phi} \left(\frac{\partial u_r}{\partial r} \right)
\end{align}

\section{Vector-Tensor Product $\left(\vgrad\mu\right)\cdot\etwiddle$}
I always forget how to do this.  Here's a nice link:  
\\
\\
https://people.rit.edu/pnveme/personal/EMEM851n/constitutive/tensors\_rect.html
\\
\\
Again, let's adopt some shorthand.  Suppose that $\bvec\equiv\left(\vgrad\mu\right)\cdot\etwiddle$ and $\mvec\equiv\vgrad\mu$.  We then have that
\begin{align}
\label{eq:vectens}
b_r &=  \left(\etwid_{rr}\,m_r + \etwid_{r\theta}\,m_\theta + \etwid_{r\phi}\,m_\phi \right) \\ =&  \frac{\partial u_r}{\partial r} \frac{\partial \mu}{\partial r} + \frac{1}{2 r} \left[r \frac{\partial}{\partial r} \left( \frac{u_{\theta}}{r}\right) + \frac{1}{r} \frac{\partial u_r}{\partial \theta} \right] \frac{\partial \mu}{\partial \theta} + \frac{1}{2 r \sintheta} \left[\frac{1}{r \sintheta} \frac{\partial u_r}{\partial \phi} + r \frac{\partial}{\partial r} \left(\frac{u_{\phi}}{r} \right) \right] \frac{\partial \mu}{\partial \phi}  \\ \\
b_\theta &=\left(\etwid_{\theta r}\,m_r + \etwid_{\theta\theta}\,m_\theta + \etwid_{\theta\phi}\,m_\phi \right) \\=& \frac{1}{2} \left[r \frac{\partial}{\partial r} \left(\frac{u_{\theta}}{r} \right) + \frac{1}{r} \frac{\partial u_r}{\partial \theta} \right] \frac{\partial \mu}{\partial r} + \frac{1}{r^2} \left[\frac{\partial u_{\theta}}{\partial \theta} + u_r \right] \frac{\partial \mu}{\partial \theta} + \frac{1}{2 r^2 \sintheta} \left[\sintheta \frac{\partial}{\partial \theta} \left(\frac{u_{\phi}}{\sintheta} \right) + \frac{1}{\sintheta} \frac{\partial u_{\theta}}{\partial \phi} \right] \frac{\partial \mu}{\partial \phi}\\ \\ 
b_\phi &=  \left(\etwid_{\phi r}\,m_r + \etwid_{\phi\theta}\,m_\theta + \etwid_{\phi\phi}\,m_\phi \right) \\ =& \frac{1}{2} \left[\frac{1}{r \sintheta} \frac{\partial u_r}{\partial \phi} + r \frac{\partial}{\partial r} \left(\frac{u_{\phi}}{r} \right) \right] \frac{\partial \mu}{\partial r} + \frac{1}{2 r^2} \left[\sintheta \frac{\partial}{\partial \theta} \left(\frac{u_{\phi}}{\sintheta} \right) + \frac{1}{\sintheta} \frac{\partial u_{\theta}}{\partial \phi} \right] \frac{\partial \mu}{\partial \theta} + \frac{1}{r^2 \sintheta} \left[\frac{1}{\sintheta} \frac{\partial u_{\phi}}{\partial \phi} + u_r + u_{\theta} \cottheta \right] \frac{\partial \mu}{\partial \phi} 
\end{align}

\end{document}
